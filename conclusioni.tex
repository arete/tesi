\chapter*{Conclusioni}
\addcontentsline{toc}{chapter}{Conclusioni}

L'esperimento OPERA nasce per misurare la probabilit\`a di oscillazione
$\nu_{\mu} \rightarrow \nu_{\tau}$ nella regione dei parametri indicata
dagli esperimenti con i neutrini atmosferici. Esso si basa sulla
rivelazione del vertice di produzione e di decadimento del leptone $\tau$,
resa possibile dalla risoluzione sub-micrometrica delle emulsioni nucleari
utilizzate come sistema tracciante. Per ottenere un bersaglio con la elevata massa
che \`e necessaria, data la bassissima sezione d'urto dei neutrini e il flusso ridotto a
 causa della distanza dalla produzione, viene usata la tecnica nota come Emulsion
Cloud Chamber. In essa, i fogli di emulsione si alternano a sottili  lastre di piombo che
sostanzialmente forniscono la massa del bersaglio per il neutrino.

Le emulsioni nucleari negli ultimi 20 anni hanno conosciuto una
rinascita grazie allo sviluppo di microscopi automatici ad alta
velocit\`a che hanno reso completamente automatica quella che un tempo
era un'analisi esclusivamente visiva e pertanto faticosissima e lenta
al microscopio. Un progetto di ricerca mirato allo sviluppo di
microscopi di nuova generazione che consentissero l'analisi di
lastre di emulsioni ad altissima velocit\`a (circa 20~cm$^2$/h) \`e
partito nel 2001 e ha conseguito recentemente i risultati prefissati
in termini di velocit\`a, efficienze e purezze di tracciamento. 

Collegato a questo progetto e parallelamente ad esso, \`e in fase di
sviluppo il software di ricostruzione e analisi dei dati provenienti dal
sistema di scansione ad alta velocit\`a realizzato. In particolar modo, il
tracciamento di particelle al minimo di ionizzazione in Emulsion Cloud
Chamber \`e stato affrontato con successo mentre \`e ancora in fase di
sviluppo la procedura di localizzazione, ricostruzione e analisi dei
vertici. La tesi \`e stata mirata a tali studi. Una Emulsion Cloud
Chamber, costituita da 56 lastrine, \`e stata esposta ad un fascio di
pioni di 8 GeV prodotti dall'acceleratore PS al CERN di Ginevra.

Partendo dalle lastre pi\`u a valle relativamente al fascio di pioni, le
tracce ricostruite sono state seguite nelle lastre pi\`u a monte fino a
definire un possibile vertice di interazione dalla scomparsa su tre lastre
consecutive della traccia cercata.  Dopo aver definito un possibile
vertice, questo viene validato da una scansione aggiuntiva in cui viene
analizzato un volume di emulsione intorno al punto di arresto ottimizzato
in modo da massimizzare l'efficienza di ricostruzione di vertici al suo
interno. La procedura \`e stata applicata a 433 tracce localizzate e  misurate
nelle lastrine pi\`u a valle e inseguite per tutte le 56 lastre. Sono
stati determinati 119 punti di arresto e, intorno ad esso,
analizzati volumi di $5 \times 5$~mm$^2$ per 8 lastre consecutive.

\`E stata messa a punto, automatizzandola, la procedura di
inseguimento delle tracce in lastre successive, con un'analisi di
$\chi^2$. I punti di arresto delle tracce sono stati definiti  dalla
mancanza del segmento cercato in tre lastre consecutive, tenendo
opportunamente conto dell'inefficienza del tracciamento. Dopo la
scansione aggiuntiva nel volume circostante il punto di arresto, \`e stata
sviluppata un'analisi basata sulla tecnica del parametro di impatto, per
selezionare vertici a due o pi\`u tracce. Alcuni di questi
vertici sono anche stati validati da un'ispezione visiva al
microscopio della lastra precedente e successiva a quella di piombo
dove l'interazione \`e avvenuta. 


Tale analisi costituisce un primo importante passo verso la ricostruzione 
completamente automatizzata di vertici di interazione nella Emulsion Cloud Chamber 
dell'esperimento OPERA, secondo una procedura complessa e delicata.
Essa necessita, in particolare, di ulteriore perfezionamento dei criteri di selezione
 per ottimizzare l'efficienza di ricostruzione.

Gli algoritmi sviluppati e messi a punto con interazioni di pioni fanno
parte nel software di ricostruzione dei dati di emulsioni e saranno
interfacciati con il sistema di database in fase di sviluppo. La
metodologia messa a punto e gli algoritmi sviluppati saranno utilizzati
con i primi fasci di neutrini disponibili, in prove sul fascio NuMI al Fermilab
nel corso del 2005 e durante la presa dati dell'esperimento OPERA a
partire dal 2006.
