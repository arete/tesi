\chapter*{Introduzione}

\addcontentsline{toc}
{chapter}{Introduzione}


La fisica del neutrino \`e uno dei settori pi\`u fervidi della fisica
delle particelle elementari. C'\`e ormai chiara evidenza a favore del
fenomeno delle oscillazioni di neutrino. Tale fenomeno si origina solo
se autostati di massa e di interazione debole non coincidono e se le
masse non sono degeneri. L'evidenza di oscillazione \`e stata ottenuta da
esperimenti con neutrini naturali (solari e atmosferici). La sua conferma e studio approfondito mediante esperimenti con fasci artificiali fanno parte di una fase di sperimentazione ora agli inizi. Queste ricerche aprono la strada a
nuova fisica oltre il Modello Standard nel settore leptonico.

L'evidenza di oscillazione nello studio dei neutrini atmosferici
necessita la conferma della transizione $\nu_{\mu} \rightarrow
\nu_{\tau}$ attraverso l'apparizione $\nu_{\tau}$ in un
fascio praticamente puro di neutrini muonici. Questa \`e la motivazione scientifica
dell'esperimento OPERA nel quale si inquadra questa tesi di
laurea. OPERA utilizzer� un fascio di neutrini muonici prodotti al CERN
(Ginevra) e rivelati al Gran Sasso. Il fascio \`e detto CNGS. La rivelazione dei $\nu_{\tau}$
avviene attraverso l'osservazione diretta della produzione e
successivo decadimento del leptone $\tau$ prodotto in interazioni di corrente carica del $\nu_{\tau}$. Questa osservazione \`e possibile grazie alla
risoluzione spaziale sub-micrometrica delle emulsioni nucleari. In OPERA, tali
rivelatori sono usati come rivelatori traccianti, intervallati con
piombo a formare una struttura nota come \emph{Emulsion Cloud
Chamber}, che permette la realizzazione di un bersaglio attivo di grande massa. L'utilizzo del piombo consente di raggiungere la massa
necessaria per il bersaglio dei neutrini e di rivelare al contempo
sciami elettromagnetici e angoli di deflessione coulombiana multipla
utili per misure cinematiche a supporto di quelle topologiche.

La tecnica delle emulsioni nucleari ha portato a importanti risultati in fisica delle particelle elementari e ha conosciuto recentemente
una vera rinascita grazie all'automatizzazione completa dell'analisi,
che prima era esclusivamente manuale o parzialmente assistita da elaboratori. Tale automatizzazione, iniziata
negli anni '70 in Giappone, ha consentito l'analisi di diverse
centinaia di migliaia di interazioni di neutrini nell'esperimento
CHORUS al CERN alla fine degli anni '90. Con l'esperimento OPERA \`e
per\`o necessario analizzare le emulsioni ad una velocit\`a di oltre
un ordine di grandezza pi\`u elevata che in CHORUS, visto anche che
l'analisi procede quasi in tempo reale durante la presa dati. Ci\`o ha
richiesto un programma dedicato di ricerca e sviluppo mirato alla
microscopia automatica ad alta velocit\`a.

Questo programma, iniziato nel 2001, ha prodotto un microscopio in
grado di analizzare le emulsioni nucleari alla velocit\`a di circa
20~cm$^2$/h. \`E in fase di sviluppo il software di ricostruzione e
analisi dei dati provenienti da tale sistema. In questo contesto si
inserisce il lavoro di questa tesi. In particolare il
lavoro \`e consistito nello sviluppo del software di ricostruzione per
 vertici delle interazioni di pioni. Lo scopo \`e di mettere a punto le metodologie che verranno utilizzate con fasci di neutrini, in un fascio di prova al Fermilab nel 2005 e sul fascio CNGS dal CERN al Gran Sasso nel 2006.

Il lavoro di tesi \`e articolato in quattro capitoli. Nel primo si
illustrano le oscillazioni di neutrino e la relativa situazione sperimentale. 
Nel secondo si presenta l'esperimento OPERA
focalizzando l'attenzione sul bersaglio, basato sulla tecnica della Emulsion Cloud
Chamber. Nel terzo si descrive la procedura di tracciamento nelle
emulsioni, indicando la procedura di connessione dei segmenti,
l'intercalibrazione delle lastre e le efficienze del tracciamento. Nel quarto capitolo si presenta la procedura sviluppata di
ricostruzione dei vertici. Come sua applicazione, si analizzano i dati ottenuti con una Emulsion
Cloud Chamber esposta a fasci di pioni al CERN. 

