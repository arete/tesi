%BIBLIOGRAFIA TESI DANILO

\begin{thebibliography}{99}

\baselineskip=15.5pt

%======================
%CITAZIONI CAPITOLO 1 =
%======================

%Becquerel
\bibitem{A1} Rabindra N. Mohapatra and Palash B. Pal, \emph{Massive Neutrinos in Physics and Astrophisics}, World Scientific (1998).

%Pauli
\bibitem{A2} J.W. Pauli, \emph{Letter to the Physical Society of Tubingen}; riprodotta in: Brown, L.M. \emph{Physics Today} \textbf{31} (1978) 9,23.

%Fermi
\bibitem{A3} E. Fermi, \emph{Z. Physik} \textbf{88} (1934) 161.

%Reins & Cowan
\bibitem{A4} F. Reins and W. Cowan, \emph{phws. Rev.} \textbf{90} (1953) 492.

%Lederman, Schwartz e Steinberger
\bibitem{A5} Danby et al., \emph{phws. Rev. Lett.} \textbf{9} (1962) 36.

%Perl
\bibitem{A6} M. Perl, \emph{PLB} \textbf{70} (1977) 487.

%Salam, Weinberg e Glashow
\bibitem{A7} S.L. Glashow, \emph{Nucl. Phys.} \textbf{22}, 597 (1961); S. Weinberg, \emph{Phys. Rev. Lett.} \textbf{19}, 1964 (1967); 
A. Salam, \emph{Proc. of the 8 Nobel Symposium on elementary particle theory, relativistic groups and analyticity}, edited by N. Svartholm, 1969.

%Pontecorvo
\bibitem{A8} B. Pontecorvo, \emph{Journal of Experimental and Theoretical Physics}, \textbf{6} (1958) 429.

%Maki
\bibitem{A9} Z. Maki et al., \emph{Prog Theor. Phys.} \textbf{28} (1962) 870.

%Dominio sensibilita'
\bibitem{A10} Bilenky S.M., Giunti C., Grimus W., \emph{Phenomenology of Neutrino Oscillations}, (1998) hep-ph/9812360.

%Bahcall
\bibitem{A11} J.N. Bachall et al., \emph{Phys. Lett.} B433 (1998) \textbf{1}.

%Homestake
\bibitem{A12} B.T. Cleveland et al. (Homestake Collaboration), \emph{Nucl. Phys.} B47 (1995).

%Gallex/GNO
\bibitem{A13} W. Hampel et al., \emph{Phys. Lett.} B447 (1999) 127.

%Sage
\bibitem{A14} J.N. Adburashitov et al., (1999) \emph{astro-ph}/9907113.

%Kamiokande
\bibitem{A15} Y. Fukuda et al., \emph{Phys. Rev. Lett.} \textbf{82} (1999) 5194.

%SuperKamiokande
\bibitem{A16} K. Kaneyuki (Super-Kamiokande Collaboration), \emph{Determination Of Neutrino Oscillation Parameters with Atmospheric Neutrinos}, \textbf{24} (2002) 112.

%SNO
\bibitem{A17} A.B. McDonald et al., (SNO Collaboration), \emph{Direct Evidence For Neutrino Flavor Transformation From Neutral-Current Interactions In Sno}, \textbf{646}, 43 (2003).

%MSW
\bibitem{A18} John N. Bahcall, (20 agosto 1996). \emph{Solar Neutrinos: Where We Are, Where We Are Going, The Astrophysical Journal}, 467, 475-484.

%Kamland
\bibitem{A19} K. Eguchi et al., (KamLAND Collaboration), \emph{First results from KamLAND: Evidence for reactor anti-neutrino  disappearance}, \textbf{90}, 021802 (2003), (arXiv:hep-ex/0212021).


%Chooz
\bibitem{A20} M. Apollonio et al., \emph{Phys. Lett.} B, \textbf{420} (1998) 397.

%Palo verde
\bibitem{A21} J. Wolf (Palo Verde Collaboration), \emph{Results From The Palo Verde Neutrino Experiment}, \textbf{549}, 795 (2002).

%Nusex
\bibitem{A22} M. Aglietta et al., \emph{Euroophys. Lett.} \textbf{8} (1989) 611.

%Frejus
\bibitem{A23} K. Daum et al., \emph{Z. Phys.} \textbf{66} (1995) 417.

%Soudan2
\bibitem{A24} D.A. Petyt (SOUDAN-2 Collaboration), \emph{Latest Results On Atmospheric Neutrinos From Soudan 2}, \textbf{110}, 349 (2002).

%IMB
\bibitem{A25} R. Becker-Szendy et al., \emph{Phys.} D \textbf{46} (1992) 3720.

%Kamiokande
\bibitem{A26} Y. Fukuda et al., \emph{Phys. Lett.} B \textbf{335} (1994) 237.

%Super-Kamiokande
\bibitem{A27} S. Fukuda et al., (Super-Kamiokande Collaboration), \emph{The Super-Kamiokande Detector}, A \textbf{501}, 418 (2003).

%MACRO
\bibitem{A28} M. Ambrosio et al., \emph{Measurement of the atmospheric neutrino-induced upgoing muon flux using MACRO}, 2001, Dec. 11.

%Chorus
\bibitem{A29} H. Shibuya (CHORUS Collaboration), \emph{A Search For Nu/Mu - Nu/Tau Oscillation With Chorus At Cern}, textbf{59}, 277 (1997).

%NOMAD
\bibitem{A30} J. Altegoer et al., (NOMAD Collaboration), \emph{Nucl. Instr. Meth.} A \textbf{404} (1998) 96.

%K2K
\bibitem{A31} R.J. Wilkes (K2K Collaboration), \emph{New results from Super-K and K2K}, eConf \textbf{C020805}, TTH02 (2002) (arXiv:hep-ex/0212035).

%LSND
\bibitem{A32} Myungkee Sung, \emph{Final Neutrino Oscillation Result form LSND, International Journal of Modern Physics}, A vol. 16, (2001) 752-754.

%KARMEN
\bibitem{A33} C. Oehler (KARMEN Collaboration), \emph{Recent Results From Karmen2}, \textbf{549}, 758 (2002).

%MiniBoone 
\bibitem{A34} E.D. Zimmerman (BooNE Collaboration), \emph{BooNE has begun}, eConf \textbf{C0209101}, TH05 (2002) (arXiv:hep-ex/0211039).


%======================
%CITAZIONI CAPITOLO 2 =
%======================


%NuMI
\bibitem{B1} K. Bourkland, K. Roon and D. Tinsley (NuMI Collaboration), \emph{205-kA pulse power supply for neutrino focusing horns}, FERMILAB-CONF-02-122-E,presented at Power Modulator Conference, Hollywood, California, Jul 1-3, 2002.

%MINOS
\bibitem{B2} E. Buckley-Geer (MINOS Collaboration), \emph{Status of the MINOS experiment}, \textbf{503}, 122 (2001).

%CNGS
\bibitem{B3} M. Nakamura, \emph{Status Of The Cngs Experiment}, \textbf{111}, 175 (2002).

%ICARUS
\bibitem{B4} J. Rico (ICARUS Collaboration), \emph{Status of ICARUS}, arXiv:hep-ex/0205028.

%OPERA
\bibitem{B5} D. Duchesneau (OPERA Collaboration), \emph{The CERN - Gran Sasso neutrino program}, \textbf{C0209101}, TH09 (2002), (arXiv:hep-ex/0209082).

%ECC
\bibitem{B6} OPERA Progress Report, CERN-SPSC 99 (2000) \textbf{20}.

%E531
\bibitem{B7} A. Gauthier (E531 Collaboration), \emph{Charmed Particle Lifetime Measurements And Limits For Neutrino Oscillations And The Existence Of The Tau-Neutrino}, FERMILAB-THESIS-1987-12.

%DONUT
\bibitem{B8} K. Hoshino (DONUT Collaboration), \emph{Result from DONUT: First direct evidence for tau-neutrino}, prepared for 8th Asia Pacific Physics Conference (APPC 2000), Taipei, Taiwan, 7-10 Aug 2000.



%======================
%CITAZIONI CAPITOLO 4 =
%======================

%Linguaggio C
\bibitem{D1} Brian W. Kernighan and Dennis M. Ritchie, \emph{The C programming language}, II edition, Prentice Hall Software Series.

%Libreria MIL
\bibitem{D2} \emph{Matrox Imaging Library} (command reference and user guide).

%Libreria FlexMotion NI
\bibitem{D3} http://zone.ni.com/devzone/devzone.nsf/webcategories.

%Filosofia di programmazione
\bibitem{D4} B. Fadini - C. Savy, \emph{Fondamenti di informatica I}, Ed. Liguori.

%Visual C++
\bibitem{D5} David J. Kruglinski, George Sheperd and Scot Wingo, \emph{Programmare in Microsoft Visual C++}, V edition, Ed. Mondadori (1998).

%Imaging
\bibitem{D6} Ian T. Young, Jan J. Gerbrands and Lucas J. van Vliet, \emph{Fundamentals of image processing}, Koninklijke bibliotheek (1998).


%======================
%CITAZIONI CAPITOLO 5 =
%======================

%Taylor
\bibitem{E1} John R. Taylor, \emph{Introduzione all'analisi degli errori}, Zanichelli (1995)

\end{thebibliography}


